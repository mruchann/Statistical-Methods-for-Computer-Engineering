\documentclass[12pt]{article}
\usepackage[utf8]{inputenc}
\usepackage{float}
\usepackage{amsmath}
\usepackage{listings}
\usepackage{graphicx}

\usepackage[hmargin=3cm,vmargin=6.0cm]{geometry}
%\topmargin=0cm
\topmargin=-2cm
\addtolength{\textheight}{6.5cm}
\addtolength{\textwidth}{2.0cm}
%\setlength{\leftmargin}{-5cm}
\setlength{\oddsidemargin}{0.0cm}
\setlength{\evensidemargin}{0.0cm}

%misc libraries goes here

\begin{document}

\section*{Student Information } 

Full Name :  Mehmet Rüçhan Yavuzdemir \\
Id Number : 2522159 \\

\section*{Answer 1}


\subsection*{a)} 

We are asked to calculate the expected values of each different colored dice.
\begin{equation*}
    E(x) = \sum\limits_{x} x \cdot f_X(x) 
\end{equation*}

Since each die's outcomes are equally likely, we can divide 1 by the number of surfaces, and multiply by the number of occurrences in the list to find probabilities of individual outcomes.

\begin{align*}

\textbf{Blue Die:}

    E_B(x) = \sum\limits_{x} x \cdot f_X(x) = \\
    (1/6) \cdot 1 + \\
    (1/6) \cdot 2 + \\
    (1/6) \cdot 3 + \\
    (1/6) \cdot 4 + \\
    (1/6) \cdot 5 + \\
    (1/6) \cdot 6 \\
    = (1/6) \cdot (1+2+3+4+5+6) = (1/6) \cdot ((7/2)\cdot 6) = 7.2 = \textbf{3.5} \\

\textbf{Yellow Die:}
E_Y(x) = \sum\limits_{x} x \cdot f_X(x) = \\
    (3/8) \cdot 1 + \\
    (3/8) \cdot 3 + \\
    (1/8) \cdot 4 + \\
    (1/8) \cdot 8 \\
    = 3/8 + 9/8 + 1/2 + 1 = 24/8 = \textbf{3} \\

\textbf{Red Die:}
E_R(x) = \sum\limits_{x} x \cdot f_X(x) = \\
    (1/2) \cdot 2 + \\
    (1/5) \cdot 3 + \\
    (1/5) \cdot 4 + \\
    (1/10) \cdot 6 \\
    = 1 + 3/5 + 4/5 + 3/5 = \textbf{3} \\
    
\end{align*}

\subsection*{b)} 
Let X and Y be a discrete random variable that is the value of the dice, respectively single die of each color and three blue dice. We have two choices, and if we calculate the expected value of X and Y, I would choose the greater one. 
\begin{equation*}
    \begin{aligned}
        E(X) = 1 \cdot 3.5 + 1 \cdot  3 + 1 \cdot 3 = \textbf{9.5} \\
        E(Y) = 3 \cdot 3.5 = \textbf{10.5} \\
    \end{aligned}
\end{equation*}


I would choose three blue dice because the expected value of the dice is greater by our calculations.

\subsection*{c)} 
Let X and Y be a discrete random variable that is the value of the dice, respectively single die of each color and three blue dice.

If $E_Y(X) = 8$, the expected value of the yellow die is 8, then:
\begin{equation*}
    \begin{aligned}
        E(X) = 1 \cdot 3.5 + 1 \cdot  8 + 1 \cdot 3 = \textbf{14.5} \\
        E(Y) = 3 \cdot 3.5 = \textbf{10.5} \\
    \end{aligned}
\end{equation*}

Hence 14.5 $>$ 10.5, rolling dice with three different colors makes more sense now, I would choose it because the expected value of the dice is greater by our calculations. 

\subsection*{d)} 
This is a conditional probability question, we are given that the value of the die is 3, and asked to find what is the probability that the rolled die is red.

Let's define our events: \\ 

\begin{center}
    \textbf{A = rolled die is red} \\ 
    \textbf{B = value of the die is 3} \\
\end{center}
\begin{equation*}
    P\{\text{A } | \text{ B}\}    
\end{equation*}

First, look, it seems the number of 3s in the red die divided by the number of total 3s is the result, but it's wrong because outcomes are not equally likely.

We can apply Bayes Rule to this question: 

\begin{equation*}
    P\{\text{A } | \text{ B}\} = \frac{P\{\text{B } | \text{ A}\} \cdot P\{\text{A}\}}{P\{\text{B}\}}    
\end{equation*}

Since each color has equal probability in random choosing, \textbf{P\{\text{A}\} = 1/3}.

When a rolled die is red, the probability of getting 3 is \textbf{P\{\text{B} $|$ \text{A}\} = 1/5}.

The only thing we have to find is P\{B\}. 
\begin{equation*}
    P\{B\} = (P\{B\} \land P\{blue\}) \lor (P\{B\} \land P\{yellow\}) \lor (P\{B\} \land P\{red\})    
\end{equation*}

\begin{equation*}
    P\{\text{A } | \text{ B}\} = \frac{\text{P\{A\} $\land$ P\{B\}}}{P\{B\}}
\end{equation*}

\begin{equation*}
    P\{B\} =    P\{\text{A } | \text{ blue}\} \cdot P\{\text{blue}\} + \\
                P\{\text{A } | \text{ yellow}\} \cdot {P\{\text{yellow}\} + \\
                P\{\text{A } | \text{ red}\} \cdot P\{\text{red}\}
\end{equation*}

\begin{align*}
    P\{\text{A } | \text{ blue}\} \cdot P\{\text{blue}\} = 1/6 \cdot 1/3 = 1/18\\
    P\{\text{A } | \text{ yellow}\} \cdot P\{\text{yellow}\} = 3/8 \cdot 1/3 = 1/8 \\
    P\{\text{A } | \text{ red}\} \cdot P\{\text{red}\} = 1/5 \cdot 1/3 = 1/15 \\
    1/18 + 1/8 + 1/15 = \textbf{P\{B\}} = \textbf{267/1080}
\end{align*}

Hence, by the Bayes Rule, 

\begin{equation*}
    P\{\text{A } | \text{ B}\} = \frac{P\{\text{B } | \text{ A}\} \cdot P\{\text{A}\}}{P\{\text{B}\}} \\ 
    = \frac{(1/5) \cdot (1/3)}
                {(267/1080)} = \textbf{24/89}
\end{equation*}

\subsection*{e)} 
The question is quite straightforward.
\begin{align*}
There are only three cases, we will calculate probabilities with respect to equally likely outcomes in dice and apply and/or operation. \\

    Blue: 1 Yellow: 4 \\ (1/6) \cdot (1/8) = (1/48) \\

    Blue: 2 Yellow: 3 \\ (1/6) \cdot (3/8) = (1/16) \\

    Blue: 4 Yellow: 1 \\ (1/6) \cdot (3/8) = (1/16) \\

    Blue: 3 Yellow: 2 \rightarrow \text{invalid since there is no 2 in yellow die.} \\\\

    = (1/48) (1+3+3) = \textbf{7/48}
\end{align*}

\section*{Answer 2}

\subsection*{a)} 
For this question, we will use Binomial Discrete Distribution. \\\\

Let X be a random variable that represents the number of distributors of A that will offer discounts. We are looking for $P\{X \geq 4 \}$. \\
\begin{align*}
    n: \text{number of trials = 80} \\
    p: \text{probability of success = 0.025} \\
    x: \text{number of successes $\geq$ 4} \\
\end{align*}

\begin{equation*}
    f_X(x) = \binom{n}{x} p^x (1-p)^{n-x}
    = \binom{80}{x} p^x (1-0.025)^{80-x} 
\end{equation*}

If we use binomial cumulative distribution with 3 successes (in this case discounts), we got the probability of getting a maximum of 3 discounts. \\

Hence, our result is \textbf{1 - binocdf(3, 80, 0.025) = 0.1406} by Octave/MATLAB. \\

\begin{align*}
    1 - f_X(0) - f_X(1) - f_X(2) - f_X(3) \\
    1 - 0.1319 - 0.2706 - 0.2741 - 0.1827 = \textbf{0.1406}
\end{align*}

\subsection*{b)} 
For this question, we will use Poisson Discrete Distribution. \\

Let X be the discount count in 2 days.

$E(X) = 2 \cdot 80 \cdot 0.025 + 2 \cdot 1 \cdot 0.1 =$ \textbf{4.2}

\textbf{$\lambda$ = 4.2}

\begin{equation*}
    f_X(x) = \frac{e^{-\lambda}\cdot\lambda^x}{x!}
\end{equation*}

Since we defined X to be the discount count in 2 days, 1 - $f_X(0)$ is what we are looking for.

\begin{equation*}
    f_X(x) = \frac{e^{-4.2}\cdot4.2^x}{x!}
\end{equation*}
$f_X(0) = e^{-4.2} = 0.015$ \\

Hence, our result is 1 - 0.015 = \textbf{0.9850}

\section*{Answer 3}

\lstset{language=matlab}

\begin{lstlisting}

blueDice = [1 2 3 4 5 6];
yellowDice = [1 1 1 3 3 3 4 8];
redDice = [2 2 2 2 2 3 3 4 4 6];
N = 1000;

for i = 1:N

    first = [
        blueDice(randi(length(blueDice))) 
        
        yellowDice(randi(length(yellowDice)))
        
        redDice(randi(length(redDice)))
    ];
    firstTotal(i) = sum(first);

    second = blueDice(randperm(length(blueDice), 3));
    
    secondTotal(i) = sum(second);
    
end

greater = sum(secondTotal > firstTotal) / N * 100;

firstAverage = mean(firstTotal);

secondAverage = mean(secondTotal);


disp(['Average total value for first choice: ' num2str(firstAverage)])

disp(['Average total value for second choice: ' num2str(secondAverage)])

disp(['Percentage of cases where second choice is better than first choice:' 
num2str(greater) '%'])


\end{lstlisting}

\begin{figure}[h]
    \centering
    \includegraphics[width=1\textwidth]{latex.png}
    \caption{Output}
    \label{fig:image}
\end{figure}


\textbf{
        It can be seen that our calculated results are very similar to experimental results, and in 57.4\% of the experiments, selecting blue dice is more advantageous. That's why I would prefer the second choice.
    }

\end{document}
